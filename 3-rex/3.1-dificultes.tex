Les principales difficultés que j'ai pu rencontrer sont liées à trois choses : le rythme d'alternance, la grande complexité fonctionnelle liée à Mosaïque et la technologie.

\subsection{Rythme d'alternance}

    Le rythme d'alternance a été un vrai souci cette année. Le développement de Mosaïque est régulé par un processus bien défini et cadré. Les cycles de développement ne laissent aucune marge pour choisir quand les évolutions sont développées.
    
    Malheureusement, le rythme de l'alternance était toujours en décalage par rapport aux développements et mes périodes entreprises commençaient toujours à la fin de la période de développement. J'ai par conséquent presque uniquement fait de la MCO. C'est une tâche utile pour appréhender l'application et il est nécessaire de continuer à en faire pour corriger les anomalies. Mais cela devient vite fastidieux et fatigant de ne faire que cela. Cela sape vite toute motivation.
    
\clearpage
\subsection{Complexité fonctionnelle}

    Mosaïque est une application de très grande envergure. Elle gère énormément de choses. C'est de plus une application très vieille qui a vu se succéder un grand nombre d'évolutions. La complexité fonctionnelle liée au métier est extrêmement importante et est la source d'une grande majorité des anomalies relevée par le client. Les règles de gestion sont tellement nombreuses et les cas tellement variés qu'aucune personne ne peut toutes les connaître ou dire si une modification viendra briser des règles existantes.
    
    De plus, le rythme d'alternance n'aide pas. Revenir après plus d'un mois d'absence suffit pour oublier la moitié de ce que j'ai appris à la période précédente. 
    
\subsection{La technologie}

    La technologie utilisée pour développer Mosaïque est obsolète et les outils associés sont vieillissants et manque de toutes les facilités proposés par les nouvelles technologies. Notamment, l'IDE n'est plus maintenu par Microsoft depuis bien longtemps et n'est pas fait pour fonctionner avec autant de projets aussi gros simultanément. Il est très instable et cela ralentit grandement le développement. Il y a également des soucis de stabilité des composants COM+ sur les postes de développement. L'IDE semble poser des problèmes d'incompatibilités binaires et de dés-enregistrement intempestifs de DLL.

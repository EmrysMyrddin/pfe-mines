Cette année n'a pas été facile, mais tout de même riche en apprentissages. Les difficultés rencontrées sont certes gênantes, mais elles sont généralement surmontables grâce à la bonne ambiance et la cohésion de l'équipe Mosaïque dans son ensemble. Tout le monde est disponible pour apporter ses connaissances à celui qui en a besoin. Des séances de programmation par paires sont organisées sur les sujets sensibles pouvant entraîner des régressions.

De plus, je suis beaucoup suivi, mon code est toujours relu et je suis accompagné dans mes échanges avec les clients. C'est ce cadre que je recherchais en quittant mon ancienne entreprise Access France Sécurité. Je suis donc tout à fait satisfait de mon choix d'avoir sacrifié l'utilisation de nouvelles technologies très intéressante pour un bon encadrement et des processus de développement bien définis. C'est, je pense, la plus importante leçon que je tire de cette année : la technologie importe assez peu face à une bonne gestion de projet.

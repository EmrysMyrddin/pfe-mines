En travaillant sur les évolutions qui m'ont été confiées, je me suis posé une question : pourquoi le temps de développement négocié avec le client était-il si juste que cela ? 

\subsection{Possession des connaissances}

En effet, le temps alloué pour les tâches que j'ai réalisées était très court, même pour un développeur confirmé. Cela m'a dans un premier temps étonné. Capgemini est chargé du développement de cette application depuis 1998 et notre équipe possède toute la connaissance technique liée à Mosaïque et ces systèmes annexes. 

Le code de l'application fonctionne d'une manière très particulière et est très tentaculaire. Il est difficile de rentrer dans l'application sans quelqu'un qui la connaît déjà depuis longtemps. C'est pourquoi je pensais qu'il serait très difficile à la SNCF de changer aujourd'hui de prestataire et donner la charge de Mosaïque à un de nos concurrents. 

\subsection{La renégociation des contrats}

    En réalité, bien que la transition serait difficile pour le nouveau prestataire, si celui-ci propose un cout moindre que le nôtre, cela ne gênerait en rien la SNCF.
    
    Un nouvel appel d'offres est lancé tous les quatre et le contrat nous liant à la SNCF est renégocié. Rien ne force la SNCF à continuer son contrat avec nous. C'est pourquoi elle peut tout de même fortement négocier les coûts de développements.
    
\subsection{Une bonne satisfaction client}

    Une chose importante entrant tout de même en jeux est la satisfaction du client. Les services de notre pôle sont très appréciés par la SNCF. Notre plateau est d'ailleurs souvent un lieu de visite. Lorsque des clients ou des personnes d'autre service viennent visiter les locaux de Nantes, le passage par notre plateforme est un point de passage obligatoire.
    
    C'est une grande force, car le client nous fait confiance lorsque nous lui expliquons qu'une fonctionnalité va prendre plus de temps que ce qu'il pensait de prime abord. En échange, il faut que nous prenions soin de toujours estimer au mieux le temps de développement des évolutions. Une telle confiance doit être entretenue pour ne pas être perdue. Car c'est très agréable de pouvoir travailler dans ces conditions.

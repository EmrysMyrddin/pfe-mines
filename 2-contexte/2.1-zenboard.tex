\subsection{L'origine}

L'idée du Zenboard est née d'une discussion entre deux consultants nantais. Ils avaient remarqué que certaines informations avaient des difficultés à circuler dans l'agence. Par exemple, le planning des formations n'était pas facile à obtenir rapidement. On ne pouvait pas savoir qui serait présent à l'agence quand et pour faire quelle formation. En bref, il s'agit principalement de problématiques locales. 

Étant dans la salle de pause durant cette discussion, l'idée d'installer une télévision affichant ces informations en permanence a alors germé. Le but était de pouvoir afficher des informations dans un lieu de passage et de rencontre. 

\subsection{Une solution modulable}

Il est vite apparu qu'une solution modulable serait incontournable. Chaque agence doit pouvoir adapter cet outil à son propre fonctionnement et ses propres contraintes locales. Il devra donc être possible pour qui le souhaite d'ajouter un module au dashboard pour afficher les informations qu'il souhaite. C'est ce que nous appellerons des \textit{plug-ins}. Un plug-in est donc un composant permettant l'affichage d'une information. Le dashboard permettra quant à lui de placer ces différents composants à l'écran.

Ainsi, le dashboard serait totalement personnalisable. Chaque utilisateur pourrait choisir les informations à afficher, leur format, leur emplacement à l'écran et le cas échéant leur comportement. Les plug-ins pourront de plus être partagés par les différents utilisateurs afin de mutualiser les développements.

\clearpage
Rapidement, l'idée a été partagée au reste de l'agence. Il est alors apparu qu'il pourrais adresser d'autres besoins. 

En effet, Zenika devais participer peu de temps après à un salon autour des nouvelles technologies: le DevFest de Nantes. Durant ce salon, on peu assister à différentes conférences autour du mobile, du web, du cloud et de l'\gls{IdO}. Un certain nombre d'entreprise, dont Zenika, possèdent également un stand qui leur permet de se présenter au public. Pour Zenika, c'est l'occasion de se faire connaître, recruter de nouveaux talents et potentiellement de nouveaux clients.

Lors d'une réunion mensuel, les consultants ont décidé de développer une preuve de concept spécialement pour cet événement. À cause du temps impartis très court, ce dashboard aurais un nombre de plug-in fixe et leur disposition ne pourra pas être changée. Les consultant ont alors proposer une liste de plug-in intéressants et se sont réparti leur développement.

Le dashboard devais afficher quelques informations, notamment un fil Tweeter, un bandeau d'actualités ainsi qu'un planning des conférences données par les consultants nantais, mais également des plug-ins plus ludique comme un carrousel de photos ou un compteur de litre de bière servis par la tireuse également présente sur le stand.

le Zenboard se vois attribuer deux nouvelles fonctions. La première est qu'il permet alors d'assurer un premier contact avec le visiteur même si tous les consultants présents sont occupés. La seconde est qu'il sert de démonstration technique et du savoir faire des consultants et d'apporter un peu d'animation sur le stand. D'autre projets comme une borne d'arcade ont été développés dans le même but.

\clearpage
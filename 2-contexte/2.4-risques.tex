\subsection{Les attentes}

  Après avoir définit spécifiquement le projet Zenboard (devenu le projet MARCEL) et effectué une première étude du marché et des potentielles solutions existante, nous avons définit des objectifs.

  Une première version de l'application devra être fournis fin août et devra permettre :

  \begin{itemize}
    \item{La création de comptes utilisateurs (via des comptes Google, Github, Facebook ou autre)}
      \item{La visualisation de la liste des Dashboards par un utilisateur}
      \item{La création via un éditeur visuel de Dashboard de manière ergonomique (glissé déposé de plug-ins)}
      \item{L'utilisation de plug-in imbriqués (cf. \textit{\fullref{sec:3.1-besoin}})}
      \item{L'affichage de la liste des plug-ins disponibles}
      \item{La possibilité de développer de nouveaux plug-in}
      \item{Proposer quelques plug-ins de base montrant le potentiel du projet}
  \end{itemize}

  La définition de ses objectifs est détailé dans la \fullref{sec:3.1-besoin}

  Mais le projet ne s'arrêtera pas fin août. S'il sera dans un premier temps utiliser par l'agence de Nantes, puis étendu à tous Zenika, il a pour finalité de devenir produit édité par Zenika. Il pourra alors être vendu à des clients sous formes de \gls{SaaS} ou même sous forme physique avec un Raspberry Pi pré-installé par nos soins. Plus d'information autour du modèle d'affaire sont disponible dans le \fullref{sec:economie}

\subsection{Les risques}
  \label{sec:risques}

  Étant donner que la mission porte sur un projet de recherche et développement, les risques sont assez limités. Il reste toute fois quelque écueil que nous avons identifiés en début de projet afin de les éviter.

  Le premier risque est de créer un outil qui finalement existait déjà. Un variante de ce risque est de créer pour notre projet des composants complexes, notamment pour la gestion du glisser déposé de plug-in. Ce risque à été minimisé par une de marché sur les offres de dashboarding existantes. Une étude à également étee porté sur les projets open source des quels nous pourrins partir pour créer notre produit. Cela aurais permis de racourcir énormément les temps de developpements.

  Le second risque est qu'a force de se différencier des autres produits pour répondre à nos besoin est de finalement créer un produit qui ne réponds plus aux besoins de qui que ce soit en dehors de Zenika. C'est un problème que l'on tente de résoudre en demandant les conseil d'un expert en expérience utilisateur qui nous conseil sur l'ergonomie et la conception graphique de l'application. Il est également peu probable que ce cas de figure survienne car nous avons pris soin de nous inspirer des solutions existante pour certaines fonctionnalités. De plus, le dashboard est par définition entièrement personnalisable et devrais donc s'adapter à l'utilisateur si celui-ci le souhaite.

  Le dernier risque est lié à la fusion du projet Zenboard avec celui de MARCEL. En effet, l'agence de Lille est loin et je devrais donc travailler avec des collègues à distance. De plus, nos projet n'avais pas exactement le même but final. Il y a donc un risque pour que le projet diverge des besoin et des attentes d'une des deux entités. Ce risque est emplifier par la difficulté de communiquer induite par la distance entre les collaboratuers. Nous expliquons dans la \fullref{sec:distance} ce qui à été mis en place pour minimiser ce risque.
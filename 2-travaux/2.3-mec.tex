Dernièrement, j'ai eu l'occasion d'être affecté à un projet en parallèle de Mosaïque. C'est le cas de la plupart des personnes de l'équipe.

Ce projet est RefParc3 qui a pour but de gérer le matériel utilisé par la SNCF dans le cadre de la distribution de titre de transport, quel qu'il soit. Cette application regroupe en réalité les informations disponibles dans 4 référentiels mis à jour à des rythmes variés. Tout le travail consiste donc à faire correspondre ces données, les rendre disponibles pour le client et proposer des exports pour soit réintégrer les données dans les référentiels, soit fournir des factures et divers documents comptables.

Notamment, RefParc3 permet de gêner un fichier appelé GTR qui contient tous les incidents ayant donné lieu à une intervention technique de maintenance. Ces interventions sont facturées par les prestataires qui garantissent le bon fonctionnement du matériel. Le but de la GTR est de calculer le temps moyen d'indisponibilité en cas d'incident et d'ainsi appliquer des pénalités en cas de dépassement des limites prévues dans le contrat de maintenance.

C’est sur ce fichier que porte principalement ma montée en compétence en vue d'une future évolution. Il s'agit d'un ensemble de macros Excel et de procédure SQL stockées venant mettre à jour le fichier. Les macros font ensuite les calculs et génèrent les tableaux voulus par le client. Mon travail sur ce projet consistera à faire évoluer le fichier avec les nouveaux types de matériels et de contrat acquis par la SNCF.
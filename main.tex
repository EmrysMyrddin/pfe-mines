\documentclass[twoside, 12pt]{report}
\usepackage[utf8]{inputenc}
\usepackage[T1]{fontenc}
\usepackage[left=4cm,right=2.5cm,top=2.5cm,bottom=2.5cm]{geometry}
\usepackage{hyperref}
	\newcommand*{\fullref}[1]{\hyperref[{#1}]{\textit{\autoref*{#1} - \nameref*{#1}}}}
    \def\chapterautorefname{chapitre}
    \def\partautorefname{partie}
\usepackage[frenchb]{babel}
\usepackage{libertine}
\usepackage{fancyhdr}
\usepackage{wrapfig}
\usepackage{graphicx}
\usepackage{array}
\usepackage{lastpage}
\usepackage{titlesec}
\usepackage{titletoc}
\usepackage{glossaries}
\usepackage{enumitem}
\usepackage{pifont}
\usepackage{wrapfig}
\usepackage{color}

\usepackage{ulem}

\makeglossaries

\newglossaryentry{API}
{
    name=API, 
    description={\textbf{\underline{A}pplication \underline{P}rogramming \underline{I}nteface} ou inteface de programmation en français est un ensemble normalisé de classes, de méthodes ou de fonctions qui servent de façade par laquelle un logiciel offre des services à d'autres logiciels}
}

\newglossaryentry{API web}
{
    name=API web :, 
    description={API utilisable à travers le web par le protocole HTTP}
}

\newglossaryentry{RFID}
{
    name=RFID,
    description={
        \textbf{\underline{R}adio \underline{F}requency \underline{Id}entification} est une technologie permettant de mémoriser et récupérer des données à distance en utilisant des marqueurs appelés tags RFID.\\
        Les tags sont de petits objets, tels que des étiquettes autoadhésives, qui peuvent être collés ou incorporés dans des objets ou produits (des antivols dans notre cas). Les tags comprennent une antenne associée à une puce électronique qui leur permet de recevoir et de répondre aux requêtes radio émises depuis l’émetteur-récepteur.\\
        Ces puces électroniques contiennent un identifiant et éventuellement des données complémentaires
    }
}

\newglossaryentry{service web}
{
    name=Service Web,
    description=
    {
        Un service web (ou service de la toile1) est un programme informatique de la famille des technologies web permettant la communication et l'échange de données entre applications et systèmes hétérogènes.
        Il s'agit donc d'un ensemble de fonctionnalités exposées sur internet ou sur un intranet, par et pour des applications ou machines.
        Le protocole de transport est généralement HTTP(S). Le service possède également souvent une architecture \underline{\gls{REST}}
    }
}

\newglossaryentry{REST}
{
    name=REST, 
    description=
    {
        Le \textbf{\underline{R}epresentational \underline{S}tate \underline{T}ransfer} est un type d'architecture pour un système particulièrement adapté aux \underline{\gls{service web}}. Les fonctions du service sont vues comme des ressources. On peut alors ajouter, supprimer et modifier ces ressources via une \underline{\gls{URI}}
    }
}

\newglossaryentry{URI}
{
    name=URI, 
    description={\textbf{\underline{U}niform \underline{R}esource \underline{I}dentifier} ou identifiant uniforme de ressource, est une courte chaîne de caractères identifiant une ressource sur un réseau}
}

\newglossaryentry{VM}
{
    name=VM,
    description=
    {
        Une \textbf{\underline{V}irtual \underline{M}achine} ou Machine Virtuelle permet d'interpréter le code Javascript qui lui est fourni. Une VM Javascript est créée pour chaque page et pour chaque \textit{iframe} quelle contient. Le code n'a accès qu'aux données contenues dans sa propre VM
    }
}

\newglossaryentry{DOM}
{
    name=DOM,
    description=
    {
        Le \textbf{\underline{D}ocument \underline{O}bject \underline{M}odel} est un standard du W3C qui décrit une interface indépendante de tout langage de programmation et de toute plate-forme, permettant à des scripts d'accéder ou de mettre à jour le contenu, la structure ou le style de documents HTML
    }
}

\newglossaryentry{ORM}
{
    name=ORM,
    description=
    {
        L'\textbf{\underline{O}bject-\underline{R}elational \underline{M}apping} est une technique de programmation informatique qui créer l'illusion d'une base de données orientée objet à partir d'une base de données relationnelle en définissant des correspondances entre cette base de données et les objets du langage utilisé. On pourrait le désigner par "correspondance entre monde objet et monde relationnel"
    }
}

\newglossaryentry{framework}
{
    name=Framework,
    description=
    {
        Un framework est un ensemble cohérent de composants logiciels structurels, qui sert à créer les fondations ainsi que les grandes lignes de tout ou d’une partie d'un logiciel
    }
}

\setcounter{secnumdepth}{5}

\titleformat{\paragraph}
{\normalfont\normalsize\bfseries}{\theparagraph}{1em}{}
\titlespacing*{\paragraph}
{0pt}{3.25ex plus 1ex minus .2ex}{1.5ex plus .2ex}

\titleformat{\subparagraph}
{\normalfont\normalsize\bfseries}{\thesubparagraph}{1em}{}
\titlespacing*{\subparagraph}
{0pt}{3.25ex plus 1ex minus .2ex}{1.5ex plus .2ex}

\pagestyle{fancy}  

\renewcommand{\baselinestretch}{1.1}

\renewcommand{\glstextformat}[1]{\dotuline{#1}}
%\renewcommand{\glstextformat}[1]{#1*}
%\renewcommand{\glstextformat}[1]{\color{blue!70!black}\bfseries#1}

\makeatletter
\@addtoreset{chapter}{part}
\makeatother

\renewcommand{\chaptermark}[1]{%
\markboth{\thechapter.\ #1}{}}

\renewcommand{\headrulewidth}{1pt}
\fancyhead[C]{}
\fancyhead[RO,LE]{\textbf{Projet de fin d'étude : \textsc{Zenika}}}
\fancyhead[LO,RE]{\leftmark}

\renewcommand{\footrulewidth}{1pt}
\fancyfoot[LO,RE]{Valentin Cocaud}
\fancyfoot[RO,LE]{\textbf{\thepage / \pageref{LastPage}}}
\fancyfoot[CO,CE]{IMT Atlantique - Filière Ingénierie Logicielle}

\fancypagestyle{plain}{
	\fancyhf{} % clear all header and footer fields
	\renewcommand{\headrulewidth}{0pt}
    \renewcommand{\footrulewidth}{1pt}
    \fancyfoot[LO,RE]{Valentin Cocaud}
    \fancyfoot[RO,LE]{\textbf{\thepage / \pageref{LastPage}}}
    \fancyfoot[CO,CE]{IMT Atlantique - Filière Ingénierie Logicielle}
}

\itshape
\setlength{\parskip}{1ex plus 0.5ex minus 0.2ex}
\newcommand{\hsp}{\hspace{20pt}}
\newcommand{\HRule}{\rule{\linewidth}{0.5mm}}

\begin{document}

\begin{titlepage}
  \begin{sffamily}
		\begin{center}
			\textsc{\LARGE IMT Atlantique\\Filière Ingénierie Logicielle}\\[2cm]
			\begin{figure}[h]
				\begin{minipage}{0.5\textwidth}
					\centering
					\includegraphics[width=4cm]{img/minesNantes.png}
				\end{minipage}%
				\begin{minipage}{0.5\textwidth}
					\centering
					\includegraphics[width=4cm]{img/itii.jpg}
				\end{minipage}
			\end{figure}

			\vfill
			\HRule \\[0.4cm]
				{\huge \bfseries Projet de fin d'étude :\\\textsc{Création d'un dashboard modulaire}\\[0.4cm] }
			\HRule \\[2cm]

			\vfill
			\begin{minipage}{0.8\textwidth}
				\begin{center} \large
					{\textbf{Etudiant :}} Valentin \textsc{Cocaud}\\
					{\textbf{Tuteur d'entreprise :}} Gilles \textsc{Chabert}\\
					{\textbf{Maître d'apprentissage :}} Amelin \textsc{Orenge}\\
				\end{center}
			\end{minipage}

			\vfill
			
			\begin{figure}[h]
				\begin{minipage}{1\textwidth}
					\centering
					\includegraphics[width=6.9cm]{img/logo_zenika.png}
				\end{minipage}%
			\end{figure}
			 
			\LaTeX

			\vfill
			{\large Septembre 2017}
		\end{center}
	\end{sffamily}
\end{titlepage}

\vfill

\thispagestyle{empty}
\cleardoublepage
\thispagestyle{empty}
Je voudrais tout d'abord remercier tous les membres de l'équipe Mosaïque société pour m'avoir accueilli et aidé pendant cette année de formation.

Je remercie tout particulièrement \textbf {Philippe \textsc{Dehoux}}, mon maître d'apprentissage qui m'a permis de mettre en valeur bon nombre de compétences qui me seront utiles dans ma vie professionnelle.

Je remercie également \textbf {Romain \textsc{Berard}} et \textbf {Sébastien \textsc{Chenais}} pour leur patience et leur aide quotidienne dans mes taches techniques.

Enfin, je souhaite remercier \textbf {Gilles \textsc{Chabert}} pour son suivi et son soutient tout au long de cette année.

\vfill

\begin{center}
    Tous les mots \underline{soulignés} se trouvent dans le glossaire à la fin de ce rapport.
\end{center}

%\clearpage
%\thispagestyle{empty}
%\section*{\textbf{\underline{Résumé :}}}
%J'ai réalisé ma première année d'alternance chez Access France Sécurité. Cette entreprise installe des systèmes de sécurité dans divers établissements et fournit des services comme l'audit de sécurité ou des contrats de maintenance à long terme. Mon travail s'est déroulé au sein du service recherche et développement sous la tutelle de M Gabriel Emering, gérant de la société.

J'ai, dans un premier temps, dû faire un audit du système existant. Dans un second temps, pour palier aux problèmes observés, mon travail a porté sur deux applicatifs, IdServ et Id4Chrome, qui remplacent les logiciels d'origine.

%\section*{\textbf{\underline{Abstract :}}}
%I realized my first year of alternation at Access France Sécurité. This company installs security systems in various establishments and supply services as the safety audit or the long-term service contracts. My work took place within the research and development department under the supervision of M Gabriel Emering, manager of the company.

I had to, at first, make an audit of the existing system. For landing in the observed problems, my work concerned two application softwares, IdServ and Id4Chrome, which replace the originals softwares.

\clearpage
\thispagestyle{empty}
\setcounter{tocdepth}{2}


\tableofcontents
\thispagestyle{empty}

\titlecontents{section}
	[0pc] % un peu d'espace vertical avant.
	{}% place dans la page
	{\textsc{\textbf{ \thecontentslabel}} \large}% mise en forme du titre.
	{\textsc{\textbf{ \thecontentslabel}} \large} % mise en forme du titre quand il n'y a pas de label.
	{\dotfill \contentspage}% l'espace est complété par des points, et on indique le numéro de la page
	[\addvspace{.0001pc}]% un peu d'espace vertical après
\thispagestyle{empty}

\cleardoublepage
\setcounter{page}{1}
\section*{\textbf{Introduction}}
\label{sect:Introduction}
\addcontentsline{toc}{section}{\textbf{Introduction}}
J'ai réalisé ma première année d'alternance chez Access France Sécurité. Cette entreprise installe des systèmes de sécurité dans divers établissements et fournit des services comme l'audit de sécurité ou des contrats de maintenance à long terme. Mon travail s'est déroulé au sein du service recherche et développement sous la tutelle de M Gabriel Emering, gérant de la société.

J'ai, dans un premier temps, dû faire un audit du système existant. Dans un second temps, pour palier aux problèmes observés, mon travail a porté sur deux applicatifs, IdServ et Id4Chrome, qui remplacent les logiciels d'origine.

\cleardoublepage
\setcounter{page}{1}
\section*{\textbf{Abstract}}
\label{sect:Abstract}
\addcontentsline{toc}{section}{\textbf{Abstract}}
I realized my first year of alternation at Access France Sécurité. This company installs security systems in various establishments and supply services as the safety audit or the long-term service contracts. My work took place within the research and development department under the supervision of M Gabriel Emering, manager of the company.

I had to, at first, make an audit of the existing system. For landing in the observed problems, my work concerned two application softwares, IdServ and Id4Chrome, which replace the originals softwares.

\part{L'entreprise et le contexte du projet}

    \chapter{Présentation de l'entreprise}
        Capgemini S.A. est une des plus importantes ESN dans le monde. Elle fournit des conseils, des technologies et de la sous-traitance de service aux clients souhaitant améliorer leur performance et leur position concurrentielle par les nouvelles technologies. Elle est présente dans quarante-quatre pays pour plus de cent vingt-cinq mille employés.

        \section{Zenika}
            Zenika est une \underline{\gls{ESN}} créée


        \section{Zenika Nantes}
            \input{1-presentation/1.2-nantes}

        \section{Stratégie}
            \input{1-presentation/1.3-strategie}

        \section{Mon positionement dans l'entreprise}
        	\input{1-presentation/1.4-positionement}

    \chapter{Le contexte de la mission}
        La mission qui m'a été confié est la réalisation du projet Zenboard. Il s'agit d'une solution modulaire de création et d'affichage de dashboard\footnotemark. 

\footnotetext{Dans le cadre de ce rapport, un dashboard est un affichage en plein écran (de préférence sur une télévision) d'information. Il ne s'agit pas de dashboard dans le sens d'intelligence d'affaire. Les information affichable peuvent être de n'importe quelle source et nature (texte, photo, graphique, vidéo)}

        \section{Le Zenboard}
        	\subsection{L'origine}

L'idée du Zenboard est née d'une discussion entre deux consultants nantais. Ils avaient remarqué que certaines informations avaient des difficultés à circuler dans l'agence. Par exemple, le planning des formations n'était pas facile à obtenir rapidement. On ne pouvait pas savoir qui serait présent à l'agence quand et pour faire quelle formation. En bref, il s'agit principalement de problématiques locales. 

Étant dans la salle de pause durant cette discussion, l'idée d'installer une télévision affichant ces informations en permanence a alors germé. Le but était de pouvoir afficher des informations dans un lieu de passage et de rencontre. 

\subsection{Une solution modulable}

Il est vite apparu qu'une solution modulable serait incontournable. Chaque agence doit pouvoir adapter cet outil à son propre fonctionnement et ses propres contraintes locales. Il devra donc être possible pour qui le souhaite d'ajouter un module au dashboard pour afficher les informations qu'il souhaite. C'est ce que nous appellerons des \textit{plug-ins}. Un plug-in est donc un composant permettant l'affichage d'une information. Le dashboard permettra quant à lui de placer ces différents composants à l'écran.

Ainsi, le dashboard serait totalement personnalisable. Chaque utilisateur pourrait choisir les informations à afficher, leur format, leur emplacement à l'écran et le cas échéant leur comportement. Les plug-ins pourront de plus être partagés par les différents utilisateurs afin de mutualiser les développements.

\clearpage

        \section{Devfest 2016 : Première expérimentation }
            Rapidement, l'idée a été partagée au reste de l'agence. Il est alors apparu qu'il pourrais adresser d'autres besoins. 

En effet, Zenika devais participer peu de temps après à un salon autour des nouvelles technologies: le DevFest de Nantes. Durant ce salon, on peu assister à différentes conférences autour du mobile, du web, du cloud et de l'\gls{IdO}. Un certain nombre d'entreprise, dont Zenika, possèdent également un stand qui leur permet de se présenter au public. Pour Zenika, c'est l'occasion de se faire connaître, recruter de nouveaux talents et potentiellement de nouveaux clients.

Lors d'une réunion mensuel, les consultants ont décidé de développer une preuve de concept spécialement pour cet événement. À cause du temps impartis très court, ce dashboard aurais un nombre de plug-in fixe et leur disposition ne pourra pas être changée. Les consultant ont alors proposer une liste de plug-in intéressants et se sont réparti leur développement.

Le dashboard devais afficher quelques informations, notamment un fil Tweeter, un bandeau d'actualités ainsi qu'un planning des conférences données par les consultants nantais, mais également des plug-ins plus ludique comme un carrousel de photos ou un compteur de litre de bière servis par la tireuse également présente sur le stand.

le Zenboard se vois attribuer deux nouvelles fonctions. La première est qu'il permet alors d'assurer un premier contact avec le visiteur même si tous les consultants présents sont occupés. La seconde est qu'il sert de démonstration technique et du savoir faire des consultants et d'apporter un peu d'animation sur le stand. D'autre projets comme une borne d'arcade ont été développés dans le même but.

\clearpage

        \section{Fusion avec le projet MARCEL}
            Durant la phase d'étude et de recueil du besoin\footnotemark, il est apparu qu'une autre agence de Zenika développais un projet similaire au Zenboard.\\

\footnotetext{Une description approfondie de cette phase est présente dans la seconde partie de ce rapport}

\subsection{MARIE : L'agence connectée et pilotée}

L'agence de Lille développe depuis sa création il y a 3 ans le projet de mettre en place des locaux entièrement connectés. Il se nomme MARIE\footnote{Module d'Administration Régissant les Interfaces Electroniques} et doit permettre de régir les différents objets connectés dispersés dans l'agence. Cela peut aller d'une sonde de température à la gestion de la lumière en passant par le machine à café. MARIE se base sur une gestion des \glspl{ontologies des objets} et sur une notion d'\glspl{objets virtuels}.

Le projet s'est vraiment concrétisé il y a un an quand l'équipe lilloise a commencé la fabrication de quelques objets connectés. Par exemple, l'agence est éuqipé d'un capteur d'humidité et de température, d'une enceinte connecté ou encore d'une cible de fléchette changeant la musique lorsque l'on joue. Cela étant, le véritable développement de MARIE commence au moment même ou j'écris ses lignes.

\subsection{MARCEL : Le miroir connecté}

TODO: INTEGRER UNE PHOTO DU DASHBOARD

C'est dans ce projet de locaux connectés que s'inscrit MARCEL\footnote{Miroir À Reflet Connecté Et Ludique}. Il est né du regroupement de 2 consultants qui avaient développé sur leur temps libre un miroir connecté et qui ont décidé de les fusionner pour en créer un projet pour l'entreprise. Le but est de proposer un affichage de différentes informations sur l'agence et son environnement sous la forme d'un miroir placé à l'entrée de l'agence. Il peut par exemple afficher la météo, la température au sein de l'agence ou encore la disponibilité des vélos libre-service de Lille autour des locaux.

Il s'agit d'un miroir sans tain derrière lequel un écran est placé et relié à un Raspberry Pi\footnotemark. La surface reste donc réfléchissante tout en laissant passer la lumière de l'écran. Il s'agit donc d'un objet connecté qui sera géré par MARIE. C'est également de MARIE qu'il recevra les données des autres objets connectés de l'agence. Il peut également envoyer des instructions via une commande vocale. C'est finalement une sorte de panneau de contrôle de l'agence.

\footnotetext{Un Raspberry Pi est un micro-ordinateur de faible puissance. Il est très utilisé dans le monde du "fait maison" de raison de son faible prix (~50€) et sa taille très réduite ($5 cm^2$). Il n'est pas très puissant, mais suffit pour une utilisation courante (Web, Bureautique, Vidéo ...) sous Linux.}

Si la première version de MARCEL fonctionne et propose déjà d'afficher diverses informations sur le miroir, son utilisation n'est toute fois pas aisée. Le déploiement de la solution est complexe et non documenté. Il est impossible de changer la disposition des informations affichées ni d'ajouter de nouvelles sources de données facilement. Il faut pour cela obligatoirement passer par une compilation de l'application et de tous ses composants.

\subsection{MARCEL 2.0 : Le miroir connecté modulaire et configurable}

Cette année, un sujet de stage a été proposé pour remanier entièrement le projet MARCEL. Le but est d'obtenir un affichage configurable et modulaire. Chaque information serait alors affichée par un plug-in et sa position pourra être configuré dans un fichier de configuration. Il suffira alors d'ajouter les plug-ins souhaités, de modifier le fichier de configuration et de relancer l'application pour voir les changements pris en compte.

Le comportement attendu pour cette nouvelle version est en réalité très proche de celui du Zenboard. C'est ce que nous avons pu constater à la fin de la phase d'étude et de recherche pour le Zenboard.

\subsection{Des doublons dans les projets}

C'est un point qui n'a pas manqué de me surprendre: comment deux sujets de stages aussi similaires ont pu être proposés sans que cela soit relevé ?

Ce doublon de sujet a été relevé grâce à mon tuteur d'apprentissage qui a demandé si l'idée du Zenboard intéressait d'autres agences. Sans cela, les deux projets auraient très bien pus être développés en parallèle.
\subsubsection{La liberté d'organisation}

En réalité, si chaque agence appartient au groupe Zenika, elle possède tout de même une très grande liberté quand à son activité et son organisation. Nous sommes liés par une vision, une stratégie et une passion commune, mais la façon de l'appliqué est laissé à la discrétion de chacun. La hiérarchie et la structure des agences sont définis par le groupe, mais pour le reste, chaque agence peu redéfinir ses propres processus et sa propre organisation. Par exemple, à Nantes, le recrutement est différent de celui du reste du groupe. Il se passe en 3 entretiens pour que le candidat s'entretienne avec le directeur d'agence (le manager), deux consultants et le directeur technique nantais. La sélection est en plus très stricte puisque seul 5\% (sur plus de 150) des personnes ayant passé les trois entretiens sont effectivement embauchés.

\subsubsection{La liberté d'entreprendre}

Cela découle d'une volonté assumée du président général de laisser le maximum de libertés aux individus. Le groupe peut ainsi se nourrir des idées de chacune des agences tandis que les agences profitent des services supports du groupe. Cette volonté se ressent également beaucoup au niveau des consultants. Tout le monde a le droit d'apporter une idée ou un projet, proposer une amélioration dans l'organisation, ou tout simplement émettre une critique ou son mécontentement. C'est un cadre de travail très humain et organique qui est propice au développement d'un fourmillement de projets internes.

\subsubsection{Le chaos organisé}

L'un des désagréments de cette ouverture d'esprit est en revanche une difficulté pour avoir une vision des projets propres à chaque agence. La vision des projets va en effet dépendre de la volonté de son créateur de le faire connaître au reste du groupe. Il peut par exemple considérer que son projet ne vaut pas encore la peine d'être montré ou tout simplement ne pas avoir le temps de gérer cette communication chronophage dans un emploi du temps déjà chargé.

C'est d'ailleurs en partie pour tenter de faciliter cette communication interagence que le projet Zenboard est né. Cela fait partie des objectifs à long terme du projet. C'est également, pour moi, ce qui explique l’enthousiasme que l'on observe aujourd'hui autour des dashboards et plus généralement de la \textit{Data visualization}. 

\subsubsection{La productivité par l'information et la motivation}

Les entreprises grossissent et se complexifie, les niveaux de hiérarchie se multiplient, les distances entre les collaborateurs s'allongent, les acteurs se multiplient. Il y a donc aujourd'hui un enjeu de plus en plus grand de communication afin de faire redescendre l'information à tous les employés de l'entreprise. C'est un besoin né de la demande de transparence face à l'opacité de la hiérarchie, mais aussi par une volonté des directions d'impliquer les employer.

En effet, le management par la carotte et le bâton a fait son temps. Nous savons maintenant que les facteurs les plus déterminants dans la productivité sont le bien-être et la motivation (puisque l'un va rarement sans l'autre). C'est pourquoi il est aujourd'hui important d'impliquer les employés, les responsabiliser, leur rendre leur individualité et leur donner les clefs pour qu'il puisse, par eux-mêmes, trouver la motivation qui leur est nécessaire.

\clearpage


        \section{Les attentes et risques}
            \subsection{Les attentes}

Après avoir définit spécifiquement le projet Zenboard (devenu le projet MARCEL) et effectué une première étude du marché et des potentielles solutions existante, nous avons définit des objectifs.

Une première version de l'application devra être fournis fin août et devra permettre :

\begin{itemize}
	\item{La création de comptes utilisateurs (via des comptes Google, Github, Facebook ou autre)}
    \item{La visualisation de la liste des Dashboards par un utilisateur}
    \item{La création via un éditeur visuel de Dashboard de manière ergonomique (glissé déposé de plug-ins)}
    \item{L'utilisation de plug-in imbriqués (cf. \textit{\fullref{sec:3.1-besoin}})}
    \item{L'affichage de la liste des plug-ins disponibles}
    \item{La possibilité de développer de nouveaux plug-in}
    \item{Proposer quelques plug-ins de base montrant le potentiel du projet}
\end{itemize}

La définition de ses objectifs est détailé dans la \fullref{sec:3.1-besoin}

Mais le projet ne s'arrêtera pas fin août. S'il sera dans un premier temps utiliser par l'agence de Nantes, puis étendu à tous Zenika, il a pour finalité de devenir produit édité par Zenika. Il pourra alors être vendu à des clients sous formes de \gls{SaaS} ou même sous forme physique avec un Raspberry Pi pré-installé par nos soins. Plus d'information autour du modèle d'affaire sont disponible dans le \fullref{sec:economie}

\subsection{Les risques}

  - La fusion avec le projet Marcel peu faire diverger le projet des besoin d'une des deux entités + attentes
  - Investir du temps pour un projet qui tombe à l'eau
  - C'est un projet que l'on veut mener à bout, but de diversifier le métier de zenika vers l'édition.
  - Obtenir un produit qui finalement existe déjà (amoindri par l'étude)
  - Obtenir un produit qui ne répond plus aux attentes des potentiels clients.
  
Étant donner que la mission porte sur un projet de recherche et développement, les risques sont assez limités. Il reste toute fois quelque écueil que nous avons identifiés en début de projet afin de les éviter.

Le premier risque est de créer un outil qui finalement existait déjà. Un variante de ce risque est de créer pour notre projet des composants complexes, notamment pour la gestion du glisser déposé de plug-in. Ce risque à été minimisé par une de marché sur les offres de dashboarding existantes. Une étude à également étee porté sur les projets open source des quels nous pourrins partir pour créer notre produit. Cela aurais permis de racourcir énormément les temps de developpements. 

Le second risque est qu'a force de se différencier des autres produits pour répondre à nos besoin est de finalement créer un produit qui ne réponds plus aux besoins de qui que ce soit en dehors de Zenika. C'est un problème que l'on tente de résoudre en demandant les conseil d'un expert en expérience utilisateur qui nous conseil sur l'ergonomie et la conception graphique de l'application. Il est également peu probable que ce cas de figure survienne car nous avons pris soin de nous inspirer des solutions existante pour certaines fonctionnalités. De plus, le dashboard est par définition entièrement personnalisable et devrais donc s'adapter à l'utilisateur si celui-ci le souhaite.

Le dernier risque est lié à la fusion du projet Zenboard avec celui de MARCEL. En effet, l'agence de Lille est loin et je devrais donc travailler avec des collègues à distance. De plus, nos projet n'avais pas exactement le même but final. Il y a donc un risque pour que le projet diverge des besoin et des attentes d'une des deux entités. Ce risque est emplifier par la difficulté de communiquer induite par la distance entre les collaboratuers. Nous expliquons dans la \fullref{sec:distance} ce qui à été mis en place pour minimiser ce risque.

\part{Le projet}

    \chapter{Étude et conception}
    	La première phase du projet fût de recueillir le besoin et définir clairement les objectifs. À partir de ces informations, j'ai pu étudier les solutions existantes sur le marcher et en tirer une première conception de l'architecture.

Nous allons parler ici du projet MARCEL et des besoins dégagé par les deux agences Nantes et Lille.
        
       	\section{Recueil du besoin et objectifs}
        	\label{sec:3.1-besoin}

La définition du besoin est une étape très importante dans la vie du projet puisqu'elle va définir les objectifs à cours termes et l'orientation du développement.

Pour cela, un document à été créer\footnote{cf. Annexe A} définissant les différentes fonctionnalité voulues mais surtout le vocabulaire qui sera utiliser pour identifier les différentes partie du système. C'est ce document que nous avons ensuite raffiner avec l'équipe lilloise pour incorporer leur propres besoin.

\subsection{Les Médias}

  Un dashboard est un écran qui permet d'afficher des informations. C'est ce qui est afficher sur une télévision. Il à été décidé conjointement avec Lille de renomer ce composant en Media pour clarifier le faites que MARCEL ne sert pas forcément qu'a afficher des chiffres et des courbes mais toute sorte de médias.

  Il est consituer d'une grille sur laquelle sont affichés des plug-ins définits par l'utilisateur. Un Media contient la liste de se plug-ins et de leur configuration. Chaque utilisatuer pour alors créer autant de Media qu'il souhaite.

\subsection{Les Plug-ins}

  Un plug-in permet d'afficher des information sur un media. Il porte un nom unique qui l'identifie et un cerains nombre de propriétés que l'utilisateur peut paramétrer pour définir le comportement du Plug-in. La liste de ces propriétés et leur rôles est totalement libre et est définit par le développeur du plug-in.

  Chaque plug-in est composé de deux parties distincts, une partie graphique (front-end) et une partie serveur (back-end)

  \subsubsection{Partie \gls{front-end}}

    Il s'agit de l'interface graphique du plugin. C'est la brique visuel qui sera affiché par le Media. Il possède une taille et une position sur la grille en plus des propriétés définit par le développeur. Cette partie est obligatoirement réalisé en Javascript et exécuté par le navigateur du client, certaines tâche sont donc impossibles à réalisé, comme la connection à une base de donnée par exemple.

  \subsubsection{Partie \gls{back-end}}

    Il s'agit d'un programme qui sera exécuté en tache de fond sur le serveur de MARCEL. Elle n'est pas obligatoire, beaucoup de plug-ins n'en aurons pas besoin. Son but est de palier aux limitations du navigateur et permettre au plug-in d'éxécuter n'importe quel programme. Il est donc possible de créer un programme se connectant à une base donnée et renvoyant au front-end les informations dont il à besoin. Cette partie peut être réaliser dans n'importe quelle technologie. La seule limitation êtant que les fonctions doivent être exposé par une \gls{API web} contenue dans un \gls{conteneur} \gls{docker}.

  \subsubsection{Les conteneurs}

    Il s'agit d'un type spécial de plug-in. Un plug-in devient un conteneur lorsqu'il possède au moins une propriété qui contient une liste de plug-ins. Les sous plug-ins ne serons alors pas affiché par le Media directement mais par le conteneur. Cela permet à dévelopeur d'avoir un contrôle sur l'affichage des plug-ins. Quelques exemples de conteneurs de base ont déjà été définit et serons livré avec MARCEL :

    \begin{itemize}
      \item{Rotation Container : Le conteneur de rotation permet d'afficher des plug-ins à tour de rôle. L'utilisateur poura définir un temps d'affichage, par exmeple 30 secondes, et une liste de plug-ins. Le conteneur affichera alors chaque plug-ins de liste pendant 30 secondes avant d'afficher le suivant.}
      \item{Conditional Container: Le contenur à condition permet d'afficher un plug-in celon certaines conditions. Un utilisateur pourra par exemple fournir deux plug-ins et indiquer une condition. Le conteneur affichera alors un plugin si la condition est vrai et l'autre si la condition est fausse.}
    \end{itemize}

\subsection{Le providers}

  Lorsque nous avons commencer à réfléchir aux diférents plug-ins qu'il serais interessant de devellopper, nous sommes vite arrivé à la conclusion que certaines partie du code allais être dupliqué un grand nombre de fois. Par exemple, si un utilisateur veux créer un Media lui permettant de suivre l'évolution de son compte twiter. Il voudra créer plusieurs plug-ins pour afficher ses derniers tweets, ses dernière mentions, les tweets de ses abonnements etc... Seuleument il sera obliger de réecrir le code permettant d'accéder à l'\gls{API} et sera obliger de configurer chaque plug-in avec ses informations de connexions.

  Les providers servent à résoudre ce problème. Ce sont des programme qui serons exécutés sur le serveur de MARCEL à l'instar des back-end des plug-ins mais pouront être appelés par n'importe quel front-end.

  Dans notre exemple, le developpeur pourra créer un provider donnant accès à l'\gls{API} de twitter. Ainsi, tout les plug-ins interrogerons ce provider et n'auront plus besoin de partie back-end. L'utilisateur n'aurra qu'a installer le provider sur son Media.

  À terme, un utilisateur pourra définir des providers sur son compte et ensuite lié ceux-ci aux Media qui en ont besoin. Cela permet de facilité d'autant plus l'utilisation du système en facorisant au maximum les configurations.

\subsection{Le Back-office}

  Il s'agit de l'interface de gestion du système. C'est à partir de cette interface qu'un utilisateur pour créer et configurer des Media via un éditeur graphique. Il poura également enregistrer des nouveaux plug-ins ou de nouveaux providers. Il poura également gérer les droits d'accès aux différent Medias qu'il aura créer et les partagés aux autres utilisateurs du système.

  C'est sur cette brique que porte ma missions de projet de fin d'étude. C'est en effet la principale différence avec les solutions existante que nous détaillerons dans la prochaine section. D'autre projet permettent déjà de créer des dashboard mais aucun ne propose à la fois une grande liberté de personalisation et un outil de création graphique simple et ergonomique.

  La simplicité et l'ergonomie sont d'ailleurs deux de points les plus importants du projet. L'expérience utilisateur devra être soignée et devra être la plus intuitive possible pour que toute personnes, quelque soit son bagage technique, puisse utiliser le logiciel. Le second point d'attention devra être à l'expérience de développeur. Le developpement de plug-in devra être simple et agréable à prendre en main. Cela à pour but de pousser au maximum le developpement de plug-in par la comunauté.

  En effet, la réussite de la plateforme sera conditioné par la diversité de son offre de plug-in. On peu voir un exemple de ce phénomème avec les deux \glspl{framework} mobiles concurent Cordova et React Native. Cordova à été le premier et la comunauté à developper une collection de plug-in plétorique. À l'inverse, React Native est arrivé plus tard et ne possède encore aujourd'hui que très peu de plug-in compatibles. C'est ce qui freinne énormément sont addoption qui est aujourd'hui surtout possée par la grande popularité de la bibliothèque React de la quelle React Native découle.

        \section{Recherche des solutions existantes}
        	La recherche des produits existant sur le marché permet de s'assurer que le produit que l'on développe n'existe pas déjà. Mais cela sert également a trouver des idées ou des fonctionalité auquel nous n'avions pas encore penser. Cela permet également de tester les produit qui implemente nos idées afin de savoir si cela comble effectivemnet nos attente ou s'il y a des choses à adapter.

Les recherches ont aboutis en plusieurs catégories: Les plateforme \gls{SaaS} et les \glspl{framework}. Notre produit devra justement s'inspirer des deux pour obtenir le produit attendu.


\subsection{Les plateformes \gls{SaaS}}

  Les plateformes \gls{SaaS} qui propose généralement un éditeur graphique mais qui sont souvent peu personilsable. Par exemple, il est rare de pouvoir ajouter ces propres plug-ins.

  Quatres produits sont resortis de ses recherhes, soit pour le sérieux de la plateforme, soit pour une fonctionalité interessante et unique.

\subsection{Les \glspl{frameworks} }

  Les \glspl{framework} de construction de dashboard. Il permettent, via une \gls{API}, de construire un dashboard avec cette fois-ci la possibiliter de créer ses propres plug-ins. Il est tout fois nécéssaire d'avoir des bases de programation et de mettre un place la chaine de compilation correspondant au langage utilisé.

  Quelques solutions on retenu l'attention à cause de leur très bonne expérience developeur. Nous avons allons donc essayer de s'insiprer de ce qui à été fait, nottamenent par le projet mozaik.

        \section{Choix des technologies}
        	Le projet est essentielement constituée de trois briques distinctes : Le front-end qui affiche les Medias, le back-end qui expose un \gls{service web} pour gérer les Medias et les plug-ins, le back-office qui permet d'éditer les Media et configurer les plug-ins graphiquement.

Chaque briques à ses propres besoin et donc nécéssiterons différentes technologie. Même si ma mission ne portera que sur la réalisation du back-office, j'ai participer aux choix technologiques des autres briques. En effet, chaque developpeur sur le projet pourra être amener au cours de developpement d'intervenir sur toutes le briques logiciels. Il est donc important que tout le monde participe à cette discution.

\subsection{Le front-end}

  Comme nous l'avons vu dans la section précédente, le framework mozaik était très prometteur. La solution n'a toute fois pas été retenu car il semblais complexe de pouvoir charger à la volée des plugins ou la configuration du Media (la disposition des plug-ins à l'écran).

  Il a donc fallu trouver une technologie permettant de charger dynamiquement des composant graphique et la configuration du Media à afficher. Après avoir étudier les \glspl{framework} détaillés dans la section précédente, il est apparu que nous aurons besoin de créer notre propre solution pour l'affichage d'un Media.

  Nous avons avons finalement opté pour \gls{Polymer}. Il s'agit d'un \gls{polyfill}\footnotemark permettant d'implémenter le standard HTML5 de composant web. Ce standard est toujours à l'étude et n'est pas encore en version finale, ce qui explique que les navigateurs ne l'ont pas encore implémenté. Cependant, la norme est très avancé et sera bientot fixée. C'est pourquoi un groupe de dévelopeur à décider de créer le projet \gls{Polymer} pour d'ors et déjà pouvoir utilisé les composant web.

  \footnotetext{Un programme permettant d'ajouter des fonctions Javascript dans un navigateur. Cela estr très utilisé pour rendre d'ancien navigateur compatible avec les nouvelles fonctions des navigateurs modernes.}

  Le gros avantage de ce standard est qu'il propose nativement de charger à la demande des morceau d'interface. C'est exactement la fonctionalité qui nous était nécéssaire pour charger les plug-ins. Nous n'avons donc plus qu'a implémenter nous-même le chargement de la configuration des Medias.

\subsection{Le back-end}

  Le back-end devra pouvoir gérer une grande quantité de conteneur docker (pour les back-ends de plug-ins et les providers). Il devra également exposer une \gls{API} \gls{REST} qui permettra de manipuler les Medias et les plug-ins.

  La technologie choisie à donc été le langage \gls{Go} et la suite de bibliothèque \gls{Mux}. Go est un nouveau langage créé par Google et à nottamenent servis pour créer \gls{docker}. Il facilite la programtion système et nottamenent concurenciel grâce à une gestion très simplifier des processus légés. Il est également très simplist dans le sens ou les dévelopeur ont volontairement réduis le nombre de fonctionalité pour rendre le langage facil d'accès mais également excessivement performant.

  Le choix de Go est principalement basé sur la qualité de son \gls{API} pour politer \gls{docker} et sa capacité à simplifier la programation système. En effet, le back-end devra gérer un ensemble de conteneurs et leur cycle de vie mais également la comunication entre ces conteneurs et les plug-ins affichés par les Medias. Cette tâche peu s'avéré très complexe mais le système de canaux de comunication de Go simplififera grandement le programme.

  Faire ce projet en Go permet également d'avoir l'avoir l'expérience d'un projet en réell developpé dans se langage au seins de l'entreprise. Zenika à pour but de rester à la pointe de la technologie, ce projet est donc une occasion de tester cette technolgie avec un risque très modéré pour l'entreprise (cf. \fullref{sec:risques}).

\subsection{Le back-office}

  La principale fonctionalité du back-office est son éditeur graphique de Media. C'est la partie qui sera la plus complexe et qui devra être la plus egronomique et facilement utilisable. Pour cela, une application web monopage semble être la meilleur solution. En effet, il semnle illusoire aujourd'hui de proposer une application lourde à installer sur l'ordinateur du client.

  Pour réaliser une application monopage, il existe un certains nombres de \glspl{framework} différents qui fonctionent chacun avec un paradigme différent. Vous trouverez une présentation détaillé de ces \glsplframework et de leur différences dans le \fullref{sec:monopage}.

  Pour le backend, le choix s'est porté sur React pour la gestion de l'affichage et Redux pour la gestion de l'état de l'application. Le langage utilisé est Javascript bien qu'il à été décider d'également utiliser un système de typage : Flow.

  Le monde des technologies front-end (éxécuter sur le navigateur) migre doucement vers des technologies de plus en plus outilés. Il est aujourd'hui courrant de \gls{compiler} le code \gls{Javascript} mais ce n'est pas tout. Des système de typages commencent également à apparaître comme le célèbre \gls{Typescript} de Microsoft. Flow est developpé conjointement avec l'équipe de développement de React, nous avons donc décider d'essayer cette technologie. Cela permet également encore une fois de tester ce système inconnue de l'entreprise dans un projet avec un risque minimal

  Vous trouverez de plus en amples explications sur les choix technologiques du back-office dans le \fullref{sec:monopage-solution}.

        \section{Architecture de l'application}
        	\input{3-etude/3.4-architecture}

    \chapter{Organisation du projet}
    	\input{4-organisation/4.0-organisation}

        \section{L'agilité}
    		\input{4-organisation/4.1-agilite}

        \section{Le travail à distance}
    		\input{4-organisation/4.2-distance}

        \section{L'organisation par la relecture de code}
    		\input{4-organisation/4.3-relecture}

    \chapter{Le model économique}
    	\input{5-economie/5.0-economie}

        \section{L'open source}
    		\input{5-economie/5.1-opensource}

        \section{Le model SasS}
    		\input{5-economie/5.2-saas}

        \section{Le model physique}
    		\input{5-economie/5.3-physique}

\part{Questionement scientifique}

	\chapter{La problématique}
    	\input{6-problematique/6.0-problematique}
    
   	\chapter{L'histoire des application monopage}
    	\input{7-histoire/7.0-histoire}
    
    	\section{Les sites multipages}
    		\input{7-histoire/7.1-multipage}
        
        \section{Les applications web dynamique}
    		\input{7-histoire/7.2-dynamique}
        
        \section{Les applications web monopage}
    		\input{7-histoire/7.3-monopage}
        
        \section{Le cloud et les microservices}
    		\input{7-histoire/7.4-microservice}
        
	\chapter{Analyse des solutions existante}
    	\input{8-existant/8.0-existant}
    	
        \section{Les paradigmes de programmation}
    		\input{8-existant/8.1-paradigmes}
        
        \section{Les technologies}
    		\input{8-existant/8.2-technologies}
        
        \section{Les architectures technique}
    		\input{8-existant/8.3-architectures}
        
    \chapter{Proposition d'architecture}
    	\input{9-proposition/9.0-proposition}
    
    	\section{Quelle architecture technique ?}
    		\input{9-proposition/9.1-architecture}
    
    	\section{Quel paradigme ?}
    		\input{9-proposition/9.2-paradigme}
        
        \section{Quelle technologie ?}
    		\input{9-proposition/9.3-technologie}

\cleardoublepage
\vfill

\section*{\textbf{Conclusion}}
\label{sect:Conclusion}
\addcontentsline{toc}{section}{Conclusion}
Cette année riche en expériences m'a grandement fait évoluer tant sur le plan organisationnel que personnel.

En cette fin d'année, je peux en dresser le bilan. 
J'ai eu l'occasion d'apprendre beaucoup sur la tenue d'un projet et la gestion de son client. J'ai particulièrement apprécié l'accompagnement dont j'ai profité. Il s'agissait des deux points que je recherchais pour compléter mon expérience acquise l'année dernière.

J'ai pu apprendre à organiser mon travail et à garder la trace de mes actions. J'ai également appréhendé les problèmes d'intégration et de test d'une application de grande envergure et à la technologie vieillissante.

Enfin, mon contact permanent avec le client m'a permis de développer ma capacité à comprendre ses besoins et à donner des explications claires.


\cleardoublepage
\label{sect:Glossaire}
\addcontentsline{toc}{section}{Glossaire}
\printglossaries

\cleardoublepage
\section*{\textbf{Bibliographie et Webographie}}
\label{sect:Webographie}
\addcontentsline{toc}{section}{Bibliographie et Webographie}
\begin{itemize}[label=\ding{228}]
    \item \underline{RFID en ultra et super hautes fréquences UHF-SHF}, Dominique Paret.
    \item \underline{Play for Java} (www.typesafe.com/resources/e-book/play-for-java), Nicolas Leroux et Sietse de Kaper
    \item \underline{AngularJS: Up and Running}, Shyam Seshadri et Brad Green
    \item www.coffeescript.org
\end{itemize}

\cleardoublepage
\label{sect:Table des figures}
\addcontentsline{toc}{section}{Table des figures}
\listoffigures

\clearpage
  \begin{sffamily}
		\begin{center}
			\vfill
				{ \huge \bfseries Rapport de fin d'année }
			\vfill
			\begin{minipage}{0.8\textwidth}
				\begin{center} \large
					Ecole des Mines de Nantes\\
					Filière Ingénierie Logiciel\\
					$1^{ere}$ année (2014 - 2015)\\
				\end{center}
			\end{minipage}
			\vfill
			{\large Nantes, le 31 aôut 2015}
		\end{center}
	\end{sffamily}

\end{document}
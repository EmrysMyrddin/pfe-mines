Capgemini propose ses services autour de quatre métiers :

\begin{figure}[!h]
    \centering
        \includegraphics[width=10cm]{img/metiers.png}
        \caption{Les métiers}
        \label{fig:metiers}
\end{figure}

\begin{itemize}
    \item Conseil : Le conseil en stratégie et transformation a pour but d'améliorer les performances économiques et la position concurrentielle du client. Cela passe par l'identification, la structuration et la mise en place d'un projet de transformation organisationnel ou économique impactant la compétitivité et la croissance de l'entreprise. Ce service se base sur une connaissance approfondie des métiers et des processus. Il s'agit de la branche Consulting Service (CS) de Capgemini SA.
    \item Intégration de systèmes : L'intégration consiste à planifier, concevoir et développer des projets de système d'information et d'application informatique. Il s'agit de la branche Technology Services (TS).
    \item Infogérance (outsourcing) : L'outsourcing est la gestion et l'amélioration de systèmes d'information existants et ses processus métier liés, partiellement ou en totalité. Ce domaine couvre tant la maintenance d'application existante que la gestion et la modification des infrastructures (poste de travail, réseaux, sécurité, centre de données, serveurs applicatifs). Il s'agit de la branche Outsourcing Service (OS).
    \item Service Informatique de proximité : Il s'agit d'un accompagnement au jour le jour et localement d'entreprise. Le principe est de mettre à disposition des clients un savoir-faire et une expertise informatique. Il s'agit de la branche Local Professional Services (LPS) supportée par Sogeti.
\end{itemize}

L'entreprise fondée en 1967 par Serge Kampf à Grenoble s'est imposée comme l'un des leaders mondiaux de l'industrie du service informatique et du conseil. Cette croissance est le fruit d'une stratégie diversification et de développement grâce à une croissance interne et externe importante.

\subsection{1967 - 1975 : Le lancement}

    C'est à Grenoble que Serge Kampf pose les bases de Capgemini en fondant Sogeti en 1967. Entre 1973 et 1975, il fait l'acquisition des trois grandes sociétés de services CAP, IT et Gemini Computer Systems. L'entreprise, renommée Cap Gemini suite aux acquisitions, atteint alors le rang de leader européen et est présente dans vingt et un pays.
    
\subsection{1975 - 1989 : L'expansion}

    L'entreprise continue son expansion et élargit son coeur de métier vers différents domaines allant des servis intellectuels de qualité aux solutions d'investissement de capital. L'entreprise entre dans les 5 leaders mondiaux en 1989 à la suite de vastes plans de restructuration interne ainsi qu'une expansion européenne et une forte pénétration du marché américain.
    
\subsection{1989 - 1998 : Changement de stratégie}

    Le groupe cherche à développer dans le domaine du conseil en management. Pour cela, elle fait une suite d'acquisitions importantes et stratégiques comme celle de United Research ou le groupe Mac aux États-Unis.
    
    En Europe, elle continue son expansion avec de nouvelles acquisitions comme Data Logic, Hoskyns ou encore Volmac.
    
    Récemment, Capgemini a renforcé sa présence aux États-Unis et dans certains pays d'Europe par l'acquisition d'Ernest \& Young Consulting confirmant ainsi le profil international du groupe.
    
\subsection{1998 - Aujourd'hui : Stratégie de crise}

    Ces dernières années, le secteur des services informatiques a connu des difficultés qui perdurent encore aujourd'hui. Capgemini connaît une crise sévère entre 2000 et 2005 poussant la société à rencontrer son activité autour de deux métiers : l'assistance technique de proximité (Sogeti) et l'infogérance. Le groupe Capgemini SA se scinde alors en deux branches : Sogeti pour l'assistance de proximité et Capgemini pour l'infogérance et le développement.
    
    L'acquisition de Transiciiel en 2003 permet de doubler la taille de Sogeti qui représente aujourd'hui 15\% du chiffre d'affaires du groupe et est le leader européen pour l'assistance technique de proximité.
    
    Afin de réduire les coûts de développement, l'entreprise se base sur une approche appelée Rightshore. Cela consiste en un équilibre entre les capacités offshore, notamment en Inde avec plus de vingt mille collaborateurs ou encore au Maroc et en Argentine, et les ressources nearshore, notamment dans les pays comme l'Espagne ou la Pologne.
    
    Aujourd'hui, l'entreprise continue sont expansion notamment avec l'acquisition en 2014 d'Euriware (filiale d'Areva) ainsi que d'IGATE en 2015 plaçant l'Amérique du Nord comme principal marché du groupe Capgemini SA.
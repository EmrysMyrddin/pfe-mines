\makeglossaries

\newglossaryentry{plugin}
{
    name=Plugin,
    text=plugin,
    plural=plugins,
    description={}
}

\newglossaryentry{SaaS}
{
    name=SaaS (Software as a Service),
    text=SaaS,
    plural=SaaS,
    description={}
}

\newglossaryentry{conteneur}
{
    name=Conteneur,
    text=conteneur,
    plural=conteneurs,
    description={}
}

\newglossaryentry{docker}
{
    name=Docker,
    text=docker,,
    description={}
}

\newglossaryentry{Polymer}
{
    name=Polymer,
    text=Polymer,
    description={}
}

\newglossaryentry{Go}
{
    name=Go,
    text=Go,
    description={}
}

\newglossaryentry{Mux}
{
    name=Mux,
    text=Mux,
    description={}
}

\newglossaryentry{polyfill}
{
    name=Polyfill,
    text=polyfill,
    plural=polyfills,
    description={}
}

\newglossaryentry{front-end}
{
    name=Front-End,
    text=front-end,
    plural=front-ends,
    description={}
}

\newglossaryentry{back-end}
{
    name=Back-End,
    text=back-end,
    plural=back-ends,
    description={}
}

\newglossaryentry{IdO}
{
    name=IdO,
    description={L'Internet des Objets (ou IoT pour \textit{Internet of Things}) est l'extension d'internet aux objets ou lieux physiques connectés.\\\\
    Il s'agit de la dernière grande évolution d'internet souvent baptisé Web 3.0 à tort pour faire suite à l'ère du "Web Social". Le Web 3.0 est un mauvais nom car l'IdO ne se limite pas au Web. Il peu toucher n'importe quelle application basé sur internet. Il représente tout objet physique capable de communiquer des données à travers le réseau. \\\\
    Il existe aujourd'hui un grand nombre de protocole facilitant la communication de ces objets mais cela reste un des grand problèmes actuel auquel la communauté informatique doit trouver une réponse}
}

\newglossaryentry{ontologies des objets}
{
    name=Ontologie des objets,
    text=ontologie des objets,
    plural=ontologies des objets,
    description={}
}

\newglossaryentry{objets virtuels}
{
    name=Objet virtuel,
    text=objet virtuel,
    plural=objets virtuels,
    description={}
}

\newglossaryentry{ESN}
{
	name=ESN,
    description={Une Entreprise de Services du Numérique , ou anciennement Société de Services en Ingénierie Informatique (SSII), est une société de services spécialisé dans les nouvelles technologies et l’informatique. Ces entreprise peuvent offrir conseil, conception et réalisation d’outils, maintenance ou encore formation.\\\\
    Elle se distingue des éditeurs de logiciels par exemple dans le sens où elle ne propose pas de produit mais vise à accompagner un client dans la réalisation d'un projet.}
}

\newglossaryentry{API}
{
    name=API,
    description={\textbf{\underline{A}pplication \underline{P}rogramming \underline{I}nteface} ou inteface de programmation en français est un ensemble normalisé de classes, de méthodes ou de fonctions qui servent de façade par laquelle un logiciel offre des services à d'autres logiciels}
}

\newglossaryentry{API web}
{
    name=API web :,
    description={\gls{API} utilisable à travers le web par le protocole HTTP}
}

\newglossaryentry{service web}
{
    name=Service Web,
    description=
    {
        Un service web (ou service de la toile1) est un programme informatique de la famille des technologies web permettant la communication et l'échange de données entre applications et systèmes hétérogènes.
        Il s'agit donc d'un ensemble de fonctionnalités exposées sur internet ou sur un intranet, par et pour des applications ou machines.
        Le protocole de transport est généralement HTTP(S). Le service possède également souvent une architecture \underline{\gls{REST}}
    }
}

\newglossaryentry{REST}
{
    name=REST,
    description=
    {
        Le \textbf{\underline{R}epresentational \underline{S}tate \underline{T}ransfer} est un type d'architecture pour un système particulièrement adapté aux \underline{\gls{service web}}. Les fonctions du service sont vues comme des ressources. On peut alors ajouter, supprimer et modifier ces ressources via une \underline{\gls{URI}}
    }
}

\newglossaryentry{URI}
{
    name=URI,
    description={\textbf{\underline{U}niform \underline{R}esource \underline{I}dentifier} ou identifiant uniforme de ressource, est une courte chaîne de caractères identifiant une ressource sur un réseau}
}

\newglossaryentry{VM}
{
    name=VM,
    description=
    {
        Une \textbf{\underline{V}irtual \underline{M}achine} ou Machine Virtuelle permet d'interpréter le code Javascript qui lui est fourni. Une VM Javascript est créée pour chaque page et pour chaque \textit{iframe} quelle contient. Le code n'a accès qu'aux données contenues dans sa propre VM
    }
}

\newglossaryentry{DOM}
{
    name=DOM,
    description=
    {
        Le \textbf{\underline{D}ocument \underline{O}bject \underline{M}odel} est un standard du W3C qui décrit une interface indépendante de tout langage de programmation et de toute plate-forme, permettant à des scripts d'accéder ou de mettre à jour le contenu, la structure ou le style de documents HTML
    }
}

\newglossaryentry{ORM}
{
    name=ORM,
    description=
    {
        L'\textbf{\underline{O}bject-\underline{R}elational \underline{M}apping} est une technique de programmation informatique qui créer l'illusion d'une base de données orientée objet à partir d'une base de données relationnelle en définissant des correspondances entre cette base de données et les objets du langage utilisé. On pourrait le désigner par "correspondance entre monde objet et monde relationnel"
    }
}

\newglossaryentry{framework}
{
    name=Framework,
    text=framework,
    plural=framewroks,
    description=
    {
        Un framework est un ensemble cohérent de composants logiciels structurels, qui sert à créer les fondations ainsi que les grandes lignes de tout ou d’une partie d'un logiciel
    }
}
La recherche des produits existant sur le marché permet de s'assurer que le produit que l'on développe n'existe pas déjà. Mais cela sert également a trouver des idées ou des fonctionalité auquel nous n'avions pas encore penser. Cela permet également de tester les produit qui implemente nos idées afin de savoir si cela comble effectivemnet nos attente ou s'il y a des choses à adapter.

Les recherches ont aboutis en plusieurs catégories: Les plateforme \gls{SaaS} et les \glspl{framework}. Notre produit devra justement s'inspirer des deux pour obtenir le produit attendu.


\subsection{Les plateformes \gls{SaaS}}

  Les plateformes \gls{SaaS} qui propose généralement un éditeur graphique mais qui sont souvent peu personilsable. Par exemple, il est rare de pouvoir ajouter ces propres plug-ins.

  Quatres produits sont resortis de ses recherhes, soit pour le sérieux de la plateforme, soit pour une fonctionalité interessante et unique.

\subsection{Les \glspl{frameworks} }

  Les \glspl{framework} de construction de dashboard. Il permettent, via une \gls{API}, de construire un dashboard avec cette fois-ci la possibiliter de créer ses propres plug-ins. Il est tout fois nécéssaire d'avoir des bases de programation et de mettre un place la chaine de compilation correspondant au langage utilisé.

  Quelques solutions on retenu l'attention à cause de leur très bonne expérience developeur. Nous avons allons donc essayer de s'insiprer de ce qui à été fait, nottamenent par le projet mozaik.